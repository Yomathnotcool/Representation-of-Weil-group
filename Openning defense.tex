\documentclass{article}
\usepackage[a4paper,left=3cm,right=3cm]{geometry}

\usepackage{amsmath}
\usepackage{yfonts}
\usepackage{titlesec}
\usepackage{tabularx}
\usepackage{ltxtable}
\usepackage{diagbox}
\usepackage{theorem}
\usepackage{latexsym,bm}
\usepackage{amstext}%公式中加入文字
\theoremstyle{theorem}
\newtheorem{theo}{Theorem}
\newtheorem{eg}{Example}
\newtheorem{defe}{Definition}
\newtheorem{rem}{Remark}
\newtheorem{lem}{Lemma}
\newtheorem{pro}{Proof}
\newtheorem{prop}{Proposition}
\newtheorem{coro}{Corollary}
\newtheorem{conc}{Conclusion}
\usepackage{titlesec}
\setcounter{secnumdepth}{4}
\titleformat{\paragraph}
{\normalfont\normalsize\bfseries}{\theparagraph}{1em}{}
\titlespacing*{\paragraph}
{0pt}{3.25ex plus 1ex minus .2ex}{1.5ex plus .2ex}
\usepackage{multirow}
\usepackage{multirow,makecell}
\usepackage{graphicx}
\usepackage{mathtools}
\usepackage{listings}
\usepackage{float}
\usepackage{makecell}
\usepackage{colortbl}
\usepackage{xcolor}
\usepackage{array}
\usepackage{latexsym}
\usepackage{amsfonts}
\usepackage{amssymb}
\usepackage{textcomp}
\usepackage{booktabs}
\usepackage{epigraph}
\usepackage{amssymb,dsfont}
\usepackage{amsbsy}
\usepackage{enumerate}
\usepackage{tikz,mathpazo}
\usetikzlibrary{shapes.geometric, arrows}
\usepackage{flowchart}
\usepackage{tikz-cd}


\usepackage{marginnote}
\usepackage{xcolor}
\usepackage[ref]{cite}
\newcommand{\Disc}{\mathop{\mathrm{Dsic}}}
\newcommand{\End}{\mathop{\mathrm{End}}}
%\newcommand{\mathfrak}{\mathfrak{}}
\usepackage[all]{xy}
\usepackage{cases}

\newcounter{mycnt}
\setcounter{mycnt}{0}
\renewcommand\themycnt{\roman{mycnt}}
\titleformat*{\section}{\raggedright}
\title{Representation of Weil Group}
\pagestyle{plain}
\large
\date{\today }
\author{Deng Zhiyuan}
\begin{document}
\maketitle
\section{Background}
\subsection{Local Langlands}
The local Langlands conjectures are certain conjectures in the context of the Langlands program. Where the geuine Langlands correspondence concerns global fields, the local Langlands correspondence concerns local fields.

In the case of the algebraic group $GL_{n}$, the local Langlands conjectures assert a correspondence between:


\begin{tikzcd}
F-\text{semisimple Weil-Deligne representation of the Weil group of a local field}\ F \arrow[d, dashed] \\
\text{Irreducible admissible representations of}\ GL_{n}(F) \arrow[u, dashed]                         
\end{tikzcd}

generalizing local class field theory from abelian Galois group to non-abelian Galois groups.
\subsection{The reasons why we need Weil group}
When we work with a field $F$, the arithmetic of $F$ is encapsulated in the Weil group $\mathcal{W}_{F}$ of $F$: this is a topological group, closely related to the Galois group of a separable algebraic closure of $F$, but with a rather more sensitive properties. We can investigate the arithmetic via the study of continuous (in some specific sense) representations of $\mathcal{W}_{F}$ over various algebraically closed fields of characteristic zero, such as the complex field $\mathbf{C}$ or the algebraic closure $\bar{\mathbf{Q}_{l}}$ of an $l$-adic number field. Considering the complex case, the one-dimensional representations of $\mathcal{W}_{F}$ are the same as the characters (i.e. continuous homeomorphisms) $F^{\times}\rightarrow \mathbf{C}^{\times}$:this is the essence of local class field theory. Langlands proposed, in a precise conjecture, that such representations should parametrize the $n$-dimensional representations of $\mathcal{W}_{F}$ in a manner generalizing local class field theory and compatible with parallel global considerations. and more specifically, that local Langlands for $GL_1$ is the same as local class field theory.
\subsection{something more concrete to show this}
There is a map from the set of equivalence classes of irreducile admissible representations $\Pi$ of $G(A_{f})$ to the set of finite-dimensional representations of $\mathcal{W}_{F_{w}}\subset Gal(\bar{F}/F)$ where $\mathcal{W}_{F_{w}}$ denotes the Weil group of $K$: 
\begin{equation*}
   \Pi\rightarrow R_{\xi}(\Pi)=Hom_{G(A_{f})}\big(\Pi, H^{n-1}(X,\mathcal{L}_{\xi})\big).
\end{equation*}
$A_{f}$ denotes the ring of finite adeles of $\mathbf{Q}$; To every absolutely irreducible representation $\xi$ of $G$ over $\mathbf{Q}$ , there is associated a smooth $\bar{\mathbf{Q}_{l}}$-sheaf $\mathcal{L}_{\xi}$ on $X=(X_{m})_{m}$ which is a certain projective system of a projective $n-1$-dimensional $F$-schemes.
\section{Definition of Weil Group}
\subsection{Definition on Section 28 of local Langlands}
In this section, we obtain the notation from tate's "Number Theory Background".

Let $F$ be a field, $\bar{F}$ the algebraic separable closure of $F$. For each finite extension $E$ of $F$ in $\bar{F}$, let $G_{E}=Gal(\bar{F}/E)$. If $G$ is a topological group, $G^{c}$ is the closure of its commutator subgroup, and $G^{ab}=G/G^{c}$ is the maximal abelian Hausdorff quotient of $G$. We begin with the exact sequence for $E,F$ finite Galois extensions of $\mathbf{Q}_{p}$:
\begin{equation*}
    1\rightarrow \mathcal{I}_{E/F}\rightarrow Gal(E/F) \rightarrow Gal(k_{E}/k_{F})\rightarrow 0
\end{equation*}
Furthermore, we can put this sequence into this
\begin{equation*}
    1\rightarrow \mathcal{I}_{F}=Gal(\bar{F}/F_{\infty})\rightarrow \Omega_{F}=\lim\limits_{\leftarrow}Gal(E/F)\rightarrow Gal(F_{\infty}/F)\cong \lim\limits_{\xleftarrow\limits_{m\geq 1}} \mathbf{Z}/m\mathbf{Z}\cong\prod\limits_{l}\mathbf{Z}_{l}=\hat{Z}\rightarrow 0 
\end{equation*}
This is the setting for the local Weil group. We now state the existence theorem of local class field theory:
\begin{theo}
Let $E$ be a finite Galois extension of $F$. The map $E\rightarrow F^{\times}/N_{E/F}E^{\times}$ is a bijection between finite abelian Galois extensions $E$ of $F$ with fintie index open subgroups of $F^{\times}$.
\end{theo}
\begin{coro}
The local reciprocity map $\theta_{F}:F^{\times}\rightarrow W^{ab}_{F}\subset G^{ab}_{F}$.
\end{coro}

Let $_{a}{\mathcal{W}_{F}}$ denote the inverse image in $\Omega_{F}$ of the cyclic subgroup $<\phi_{F}>$ of $Gal(F_{\infty}/F)$. Thus it's the dense subgroup of $\Omega_{F}$ generated by the Frobenius elements. It is normal in $\Omega_{F}$ and it fits into an exact sequence( of abstract groups)
\begin{equation*}
    1\rightarrow \mathcal{I}_{F}\rightarrow _{a}\mathcal{W}_{F}\rightarrow\mathcal{Z}\rightarrow 0
\end{equation*}

\begin{defe}
The Weil group $\mathcal{W}_{F}$ of $F$ (relative to $\bar{F}/F$) is the topological group, with underlying abstract group $_{a}\mathcal{W}_{F}$, so that 
\begin{enumerate}
    \item $\mathcal{I}_{F}$ is an open subgroup of $\mathcal{W}_{F}$;
    \item the topology on $\mathcal{I}_{F}$, as subspace of $\mathcal{W}_{F}$, coincides with its natural topology as $Gal(\bar{F}/F_{\infty})\subset\Omega_{F}$.
\end{enumerate}
\end{defe}
Thus $\mathcal{W}_{F}$ is locally profinite, and the identity map $\tau_{F}:\mathcal{F}\rightarrow _{a}\mathcal{W}_{F}\subset\Omega_{F}$ is a continous injection.

The definition of $\mathcal{W}_{F}$ does depend on the choice of $\bar{F}/F$,but only up to inner automorphism of $\Omega_{F}$.

We write $\nu_{F}:\mathcal{F}\rightarrow\mathcal{Z}$ for the canonical map taking a geometric Frobenius element to 1 and $||x||=q^{-\nu_{F}(x)},x\in\mathcal{W}_{F}.$
\subsection{Definition from Tate}

But the definition from Tate is a little bit different: he definies the Weil group as a triple $(\mathcal{W}_{F},\phi,\{r_{E}\})$ which is constructed with four conditions:
\begin{defe}
$F$ is a local field and $\bar{F}$ is a separable algebraic closure of $F$. Let $E, E',...$ denote finite extensions of $F$ in $\bar{F}$. A Weil group for $\bar{F}/F$ is not really just a group but a triple $(\mathcal{W}_{F},\phi,\{r_{E}\})$. The first two ingredients are a topological group $\mathcal{W}_{F}$ and a continuous homomorphism $\phi: W_{F}\rightarrow G_{F}=Gal(\bar{F}/E)$ with dense image. Given $\mathcal{W}_{F}$ and $\phi$, we put $\mathcal{W}_{E}=\phi^{-1}(G_{E})$ for each finite extension $E$ of $F$ in $\bar{F}$. The continuity of $\phi$ just means that $\mathcal{W}_{E}$ is open in $\mathcal{W}_{F}$ for each $E$. and it's having dense image means that $\phi$ induces a bijection of homogenous spaces:
\begin{equation*}
    W_{F}/W_{E}\rightarrow G_{F}/G_{E}\approx Hom_{F}(E,\bar{F})
\end{equation*}
for each $E$, and in particular, a group isomorphism $W_{F}/W_{E}\approx Gal(E/F)$ when $E/F$ is Galois. The last ingredient of a Weil group is, for each $E$, an isomorphism of topological groups $r_{E}:C_{E}\rightarrow W^{ab}_{E}$, where
$C_{E}$ is 
\begin{enumerate}
    \item The multiplicative group $E^{*}$ of $E$ in the local case,
    \item the idele-class group $A^{*}_{E}/E^{*}$ in the global case.
\end{enumerate}
\end{defe}
In order to constitute a Weil group these ingredients must satisfy four conditions:
\begin{enumerate}
    \item For each $E$, the composed map

\begin{tikzcd}
C_{E} \arrow[rr, "r_{E}"] &  & \mathcal{W}^{ab}_{E} \arrow[rr, "\text{induced by}\ \phi"] &  & G^{ab}_{E}
\end{tikzcd}
    is the reciprocity law homomorphism of class field theory.
    \item Let $w\in \mathcal{W}_{F}$ and $\sigma=\phi(w)\in G_{F}$. For each $E$ the following diagram is commutative:
 
 
\begin{tikzcd}
C_{E} \arrow[dd, "\text{Induced by }\sigma"'] \arrow[rr, "r_{E}"] &  & \mathcal{W}^{ab}_{E} \arrow[dd, "\text{conjugation by } w"] \\
                                                                  &  &                                                             \\
C_{E^{\sigma}} \arrow[rr, "r_{E^{\sigma}}"]                       &  & \mathcal{W}^{ab}_{E^{\sigma}}      
\end{tikzcd}

\item For $E'\subset E$ the diagram


\begin{tikzcd}
C_{E'} \arrow[rr, "r_{E'}"] \arrow[dd, "\text{Induced by inclusion }E'\subset E"'] &  & \mathcal{W}^{ab}_{E'} \arrow[dd, "\text{transfer}"] \\
                                                                                   &  &                                                     \\
C_{E} \arrow[rr, "r_{E}"]                                                          &  & \mathcal{W}^{ab}_{E}                               
\end{tikzcd}

is commutative.

\item The natural map $\mathcal{W}_{F}\rightarrow \lim\limits_{\leftarrow_{E}}\{\mathcal{W}_{E/F}\}$ is an isomorphism of topological groups, where $\mathcal{W}_{E/F}$ denotes $\mathcal{W}_{F}/\mathcal{W}^{c}_{E}$(not $\mathcal{W}_{F}/\mathcal{W}_{E}$) and the projective limit is taken over all $E$, ordered by inclusion, as $E\rightarrow\bar{F}$. 

\end{enumerate}

This is the Tate's definition for Weil group.
\subsection{Try to make this down-to-earth}
\subsubsection{linear version}
Let $k$ be finite extension of real numbers, then
\begin{enumerate}
    \item $k\cong \mathbf{C}$, then $\mathcal{W}_{K}=K^{\times}$;
    \item $k\cong \mathbf{R}$, then $\mathcal{W}_{k}= \bar{K}^{\times}\bigcup j\bar{K}^{\times}$ for $j^{2}=-1$ and $jcj^{-1}=\bar{c},\ c\in \mathbf{C},\ j\in\{1,i,j,k\}$;
\end{enumerate}
In both case, we can get $\bar{K}^{\times}$ is a normal subgroup of Weil group $\mathcal{W}_{K}$. Then $\mathcal{W}_{K}/\bar{K}^{\times}\cong \text{Gal}(\bar{K}/K)$, giving the exact sequence:
\begin{tikzcd}
1 \arrow[r] & \bar{K}^{\times} \arrow[r] & \mathcal{W}_{K} \arrow[r] & \text{Gal}(\bar{K}/K) \arrow[r] & 1
\end{tikzcd}
For non-archimedean case, the abelianisation of Weil group is equipped with an isomorphism with $K^{\times}$. For $K\cong \mathbf{C}$, this is clear. But in the real case, this isomorphism needs something more. The commutator subgroup of $\mathcal{W}_{K}$ is of form $\frac{c}{\bar{c}}$ with $c\in\mathbf{C}$, which is the unit circle $\mathcal{S}^{1}$ of $\mathbf{C}^{\times}$.

\begin{tikzcd}
\mathcal{W}_{\mathbf{R}}/\mathcal{S}^{1} \arrow[rr]                                                            &  & \mathbf{R}_{>0}\bigcup j\mathbf{R}_{>0} \\
\mathcal{W}_{\mathbf{R}}\cong\bar{\mathbf{R}}^{\times}\bigcup j\bar{\mathbf{R}}^{\times} \arrow[u, Rightarrow] &  &                                        
\end{tikzcd}

and the isomorphism from $\mathbf{R}^{\times}$ to this sends -1 to $j$ and $x>0$ to $\sqrt{x}$. For the other point of view, the isomorphism $\mathcal{W}^{ab}_{\mathbf{R}}\rightarrow \mathbf{R}$ sends $z=x+iy\in \mathbf{C}^{\times}$ to $x^{2}+y^{2}$. In particular it doesn't depend on the choice of isomorphism $\bar{\mathbf{R}}=\mathbf{C}$. The square foot or square we used here are for compatibility of this isomorphism under finite extensions of $K$; 

Next move is we can define a norm $||w||$ on $\mathcal{W}_{K}$. 
\begin{enumerate}
    \item If $K\cong \mathbf{C}$, then $||w||=w\bar{w}$(this doesn't depend on the choice of $K\cong\mathbf{C}$).
    \item If $K\cong \mathbf{R}$, then $||w||$ is $w\bar{w}$ for $w\in\mathbf{C}^{\times}$. 
\end{enumerate}
Note that the norm is a continuous group homomorphism $\mathcal{W}^{ab}_{K}\rightarrow\mathbf{R}_{>0}$ which thus gives rise to a continuous group homomorphism $K^{\times}\rightarrow \mathbf{R}_{>0}$

Now we can talk about the representation of Weil group here:

Given a continuous map into $GL(V)$, $V$ is a finite-dimensional complex vector space.
For 1-dimensional representation, it's only finite situations here. We use those 1-dimensional representation to construct thins later.
\begin{lem}
\begin{enumerate}
    \item The only continuous map group homomorphisms $\mathbf{R}\rightarrow\mathbf{C}^{\times}$ are those of the form $x\mapsto \text{exp}(sx)$, and $s\in \mathbf{C}^{\times}$;
    \item The only continuous group homomorphisms from the unit circle $\mathcal{S}^{1}$ to $\mathbf{C}^{\times}$ are of the form $z\mapsto z^{n}$ for some $n\in\mathbf{Z}$, with distinct $n$ giving distinct homomorphisms.
    \item The only continuous group homorphisms $\mathbf{R}_{>0}\rightarrow \mathbf{C}^{\times}$ are those of the form $x\mapsto x^{s}:=\text{exp}(s\text{log}(x))$ for $s\in\mathbf{C}^{\times}$, with distinct $s$ giving distinct homorphisms.
    \item The only continous group homomorphisms $\mathbf{R}^{\times}\rightarrow \mathbf{C}^{\times}$ are of the form $x\mapsto x^{-N}||x||^{s}$ for $s\in\mathbf{C}$ and $N\in\{0,1\}$, and distinct pairs $(N,s)$ give distinct homomorphisms.
\end{enumerate}
\end{lem}

So we are now seen all the 1-dimensional representations of Weil groups. In Tate's canonical sense, if $K\cong\mathbf{C}$ then the 1-dimensional representations of $\mathcal{W}_{K}$ are all of the form $z\mapsto \sigma(z)^{-N}||z||^{s}$ with $\sigma: K\rightarrow\mathbf{C}$ in which $N\geq0,s\in\mathbf{C}$ an isomorphism,we need both isomorphism to see all the representations.When $N=0$, we don't care about which $\sigma$ we choose. This normalisation is motivated by the study of $L$-functions and $\epsilon$ factors.

By basic arguments, any continuous irreducible finite-dimensional representation of $\mathcal{W}_{\mathbf{C}}$ has an eigenvector and is hence 1-dimensional, so we deduce that we have now een all the irreducible $n$-dimensional representations of $\mathcal{W}_{\mathbf{C}}$. For $\mathcal{W}_{R}$ there are some irreducible 2-dimensional representations. The point is that if $rho $ is an irreducible representation of $\mathcal{W}_{\mathbf{R}}$ of dimension greater than 1 then the restriction of $rho$ to $\mathcal{W}_{\mathbf{C}}$ must have an eigenvector, and if it's $v$ then $v$ and $jv$ span an invariant subspace, so the dimension of $rho$ is 2, and $rho$ is induced from a character of $\mathcal{W}_{\mathbf{C}}$ is of the form $z\mapsto\sigma(z)^{N}||z||^{s}$ with $N\in\mathbf{Z}_{\geq0}$ and if inducing this 1-dimensional representation then you get a 2 dimensional representation which is irreducible if $N>0$, and reducible if $N=0$.
\begin{conc}
The irreducible representations of $\mathcal{W}_{\mathbf{R}}$ are 1-dimensional of the form $\mathcal{W}^{ab}_{\mathbf{R}}=\mathbf{R}^{\times}\rightarrow\mathbf{C}^{\times}$ via $z\mapsto z^{-N}||z||^{s}$ with $N\in\{0,1\}$ and $s\in\mathbf{C}$, and 2-dimensional induced from a character $z\mapsto\sigma(z)^{-N}||z||^{s}$ on $\mathcal{W}_{\mathbf{C}}$, with $N\in \mathbf{Z}_{>0}$ and $s\in\mathbf{C}$.
\end{conc}
\subsubsection{Mission Impossible}
\paragraph{Supplement of Definition}
Let $\mathcal{O}_{\bar{K}}$ be the ring of integers of the algebraic closure $\bar{K}$ of $K$. Every element of $Gal(\bar{K}/K)$ defines an automorphism of $\mathcal{O}_{\bar{K}}$ which reduces to an automorphism of the residue field $\bar{\kappa}$ of $\mathcal{O}_{\bar{K}}$. We get a surjective map
\begin{equation*}
     \Upsilon: Gal(\bar{K}/K)\rightarrow Gal(\bar{\kappa}/\kappa) 
\end{equation*}
whose kernel is by definition the inertia group $I_{K}$ of $K$.
\begin{defe}
Let $\mathfrak{P}$ be a prime of $K$. Then the \emph{decomposition group} and \emph{inertia group} of $\mathfrak{P}$ are defined by
\begin{enumerate}
    \item[*] $D_{\mathfrak{P}}:=\{\sigma\in Gal(\bar{K}/K)| \sigma(\mathfrak{P})=\mathfrak{P}\}$;
    \item[*] $I_{\mathfrak{P}}:=\{\sigma\in Gal(\bar{K}/K)| \sigma(\alpha)=\alpha\textbf{mod}\ \mathfrak{P}\ \text{ for all }\ \alpha\in\mathcal{O}_{K}\}$
\end{enumerate}
\end{defe}
The group $Gal(\bar{\kappa}/\kappa)$ is topologically generated by the arithmetic Frobenius automorphism $\sigma_{K}$ which sends to $x\in\bar{\kappa}$ to $x^{q}$. It contains the free abelian group $<\sigma_{K}>$ generated by $\sigma_{K}$ as a subgroup.
\begin{defe}
\emph{Krull Topology}
\begin{enumerate}
    \item Let $L/K$ be a Galois extension,write 
$$\mathcal{F}=\{F|F \text{is a subfield of }L \ s.t.\ L/F \text{is a finite Galois extension}\}.$$
We define a topology in $\text{Gal}(L/K)$ by taking as a base of open neighborhoods of 1 the family of sugroups 
$$\mathcal{N}=\{\text{Gal}(L/F)|L\in \mathcal{F}\}.$$
\item The Krull Topology on $\text{Gal}(L/K)$ is defined as follows:  A subset $X$ of $\text{Gal}(L/K)$ is open if it's empty or $X=\bigcup\limits_{i}g_{i}N_{i}$ for some $g_{i}\in G$ and $N_{i}\in\mathcal{N}$.In this case, the basis of Krull Topology is $\{gN|g\in G,N\in\mathcal{N}\}.$
\end{enumerate}
Because this is a topological group $G$, then the map $\lambda_{g}:G\rightarrow G$ given by $x\mapsto gx$ is a homeomorphism that sends 1 to $g$. 
\end{defe}

\begin{defe}
\emph{Topologically generated},which makes sense when you know \emph{Krull topology}.But it's impossible to cover all the details.

If $G$ is a topological group and $S$ is a subset of $G$, we say that $S$ topologically generates $G$ if the closure of the subgroup generated by $S$ is equal to $G$.
\end{defe}
The fixed field $\bar{K}^{I_{K}}$ of $I_{K}$ in $\bar{K}$ is $K^{nr}$, the union of all unramified extensions of $K$ in $\bar{K}$.
\begin{defe}
\emph{Fixed Field}:

If $K$ is any field, and $P$ is any group of automorphism of $K$, we define the fixed field as 
\begin{equation*}
    K^{P}:=\{x\in K|\sigma(x)=x\ \text{for all}\ \sigma\in P\}
\end{equation*}
\end{defe}
\begin{defe}\emph{Frobenius Elements}:
Let $L/K$ be extension of number field, and it's Galois extension.Let $\mathfrak{P}\in\mathcal{P}_{K}$ which does not ramify in $L$. And $\mathfrak{P}\mathcal{O}_{L}=\prod\limits^{q}_{i=1}\mathcal{P}_{i}$. So for all $i$, the inertia group $\mathcal{I}_{\mathcal{P}_{i}}=\{1\}$.
\begin{equation*}
    \mathcal{D}^{L/K}_{\mathcal{P}_{i}}\cong\text{Gal}(l_{\mathcal{P}_{i}}/k)
\end{equation*}
Where $l_{\mathcal{P}_{i}}=\mathcal{O}_{L}/\mathcal{P}_{i}$ and $k=\mathcal{O}_{K}/\mathcal{P}$.
Let's look at one of the $p'_{i}$, just call it    $\mathclap{P}$.
Then for $\text{Gal}(l/k)=<\bar{\Phi}>$ where $\bar{\phi}$ is the privileged generator. It holds $\bar{\Phi}(\bar{x})^{|k|}=\bar{x}^{q}$.

So we have a privileged generator of $\mathcal{D}^{L/K}_{\mathclap{P}}$, which is called the Frobenius endomorphism. We denote it by $(\mathclap{P},L/K)$.
\begin{equation*}
    (\mathclap{P},L/K)(x)\equiv x^{q}\mod \mathclap{P}.
\end{equation*}
\end{defe}
\begin{rem}
\emph{Frobenius Elements}:
Given a scheme $X$ is said to be of characteristic $p$ if $p\mathfrak{O}_{X}=0$. 

\begin{tikzcd}
X \arrow[d] \arrow[rd, dotted] &                              \\
Spec\mathbf{Z}                 & Spec\mathbf{F}_{p} \arrow[l]
\end{tikzcd}

So saying that a scheme is of characteristic $p$ is the same thing as saying that it can be viewed as scheme over $Spec\mathbf{F}_{p}$.

Then we define the absolute Frobenius endormorphism of $X$ as $$Fr_{X}:X\rightarrow X$$
as the morphism which is the identity on $|X|$ and the $p^{th}$-power map on $\mathcal{O}_{X}$(this really defines a morphism of sheaves of rings since $p\mathcal{O}_{X}=0$)

If $X=Spec A$ is affine, the Frobenius enfomorphism of $X$ arises from the Frobenius endomorphism $a\mapsto a^{p}$ of $A$.Th really induces the identity on $|Spec A|$.
\end{rem}


\section{Representation of Weil Group}


\begin{defe}
The representation $(\pi,V)$ is called smooth if for every $v\in V$,there is a compact open subgroup $K$ of $G$ (depending on $v$) s.t. $\pi(x)v=v$, for all $x\in K$. Equivalently, if $V^{K}$ denotes the space of $\pi(K)$-fixed vectors in $V$, then 
\begin{equation*}
    V=\bigcap\limits_{K}V^{K}
\end{equation*}
where $K$ ranges over the compact open subgroups of $G$.
\end{defe}

\begin{defe}
A smooth representation $(\pi,V)$ is called admissible if the space $V^{K}$ is finite-dimensional, for each compact open subgroup $K$ of $G$.
\end{defe}

\begin{defe}
One says that $(\pi,V)$ is $G$-semisimple if it satisfies the conditions of the proposition. 

Let $G$ be a locally profinte group, and let $(\pi,V)$ be a smooth representation of $G$. The following conditions are equivalent:
\begin{enumerate}
    \item $V$ is the sum of its irreducible $G$-subspaces;
    \item $V$ is the direct sum of a family of irreducible $G$-subspaces;
    \item any $G$-subspace of $V$ has a $G$-complement in $V$.
\end{enumerate}
\end{defe}

\paragraph{Representation Story}
\begin{lem}
Let $E/F$ be a finite separable field extension, $E\subset\bar{F}$.
\begin{enumerate}
    \item Let $\rho$ be a smooth representation of $\mathcal{W}_{F}$; then $\rho$ is semisimple iff $\rho_{E}$ is semisimple.
    \item Let $\tau$ be a smooth representation of $\mathcal{W}_{E}$; then $\tau$ is semisimple iff $Ind_{E/F}\tau$ is semisimple.
\end{enumerate}
\end{lem}
For each integer $n\geq1$, we denote by $\text{艹}^{ss}_{n}(F)$ the set of isomorphism classes of semisimple smooth representation of $\mathcal{W}_{F}$ of dimension $n$. We denote by $\text{艹}^{0}_{n}(F)$ the set of isomorphism classes of irreducible smooth representations of $\mathcal{W}_{F}$ of dimension $n$.

If $E/F$ is a finite extension, $E\subset\bar{F}$, the lemma show that we have induction and restriction maps
\begin{equation*}
    Ind_{E/F}:\text{艹}^{ss}_{n}(E)\rightarrow\text{艹}^{ss}_{nd}(F),\\
    Res_{E/F}:\text{艹}^{ss}_{n}(F)\rightarrow\text{艹}^{ss}_{n}(E),
\end{equation*}
where $d=[E:F]$.
\begin{prop}
Let $(\rho,V)$ be a smooth representation of $\mathcal{W}_{F}$ of finite dimension, and let $\Phi\in \mathcal{W}_{F}$ be a Frobeniius element. The following are equivalent:
\begin{enumerate}
    \item the representation $\rho$ is semisimple;
    \item the automorphism $\rho(\phi)\in Aut_{\mathbf{C}}(V)$ is semisimple;
    \item the automorphism $\rho(\psi)\in Aut_{\mathbf{C}}(V)$ is semisimple, for every element $\Psi$ of $\mathcal{W}_{F}$.
\end{enumerate}
\end{prop}

\end{document} 